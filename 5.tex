\section{Dualité I: formulation, relations d'exclusion, solution}


{\bf Forme géométrique}

Problème primal

$
\begin{array}{lrcr}
  \mini & c^T x\\
  & A x & \geq &  b\\
  & x   & \geq & 0
\end{array}
$

Problème dual

$
\begin{array}{lrcr}
  \maxi & b^Ty\\
  & A^T y & \leq &  c\\
  & y   & \geq & 0
\end{array}
$\\


{\bf Forme standard}



Problème primal

$
\begin{array}{lrcr}
  \mini & c^T x\\
  & A x & = &  b\\
  & x   & \geq & 0
\end{array}
$



Problème dual


$
\begin{array}{lrcr}
  \maxi & b^T y\\
  & A^T y & \leq &  c
\end{array}
$


\newpage



\begin{enumerate}

  \item \label{e5.1} Trouvez le dual du problème

    $
    \begin{array}{llcr}
      \mini & 3x_1-9x_2\\
      & 2x_1-x_2+x_3 & \geq & 3\\
      & x_1-x_2 -3x_3 & \geq & 2\\
      & x_1, x_2, x_3 & \geq & 0
    \end{array}
    $

    \begin{solution}
      $
      \begin{array}{llcr}
        \maxi & 3y_1 + 2y_2\\
        & 2y_1 + y_2 & \leq & 3\\
        & -y_1 - y_2 & \leq & -9\\
        & y_1 - 3y_2 & \leq & 0\\
        & y & \geq & 0
      \end{array}
      $
    \end{solution}


  \item Trouvez le dual du problème

    $
    \begin{array}{llcr}
      \mini & 3x_1-9x_2\\
      & 4x_1-x_2+x_3 & = & 3\\
      & x_1-x_2 -6x_3 & \geq & 2\\
      & x_1-4x_2 -6x_3 & \leq & 8\\
      & x_1, x_2 & \geq & 0
    \end{array}
    $

    \begin{solution}
      $
      \begin{array}{lrcr}
        \maxi & 3y_1 + 2y_2 + 8y_3\\
        & 4y_1 + y_2 + y_3 & \leq & 3\\
        & -y_1 -y_2 -4y_3 & \leq & -9\\
        & y_1 -6y_2 -6y_3 & = & 0\\
        & y_2 & \geq & 0\\
        & y_3 & \leq & 0

      \end{array}
      $
    \end{solution}


  \item Trouvez le dual du problème

    $
    \begin{array}{llcr}
      \mini & c^T x\\
      & a_1^T x& = & b_1\\
      & a_2^T  x & \geq & b_2\\
      & a_3^T x & \leq & b_3\\
      & x& \geq & 0
    \end{array}
    $

    \begin{solution}
      $
      \begin{array}{llcr}
        \maxi & b^T y\\
        & Ay & \leq & c\\
        & y_1 & & \text{libre}\\
        & y_2 & \geq & 0\\
        & y_3 & \leq & 0
      \end{array}
      $

      où $A = \begin{pmatrix}a_1 & a_2 & a_3\end{pmatrix}$
    \end{solution}

  \item Trouvez le dual du problème de trois variables

    $
    \begin{array}{llcr}
      \mini & c^T x\\
      & a^T x& = & b\\
      & x_1 &  & \mbox{libre}\\
      & x_2 & \geq & 0\\
      & x_3 & \leq & 0
    \end{array}
    $

    \begin{solution}
      $b$ est un scalaire

      $
      \begin{array}{llcr}
        \maxi & b y\\
        & a_1 y & = & c_1\\
        & a_2 y & \leq & c_2\\
        & a_3 y & \geq & c_3\\
        & y & & \text{libre}
      \end{array}
      $
    \end{solution}

  \item Trouvez une formulation duale du  problème

    $
    \begin{array}{lrcr}
      \mini & c^T x\\
      & A x & = &  b\\
      & x   & \geq & a
    \end{array}
    $

    où $a \geq 0$.

    \begin{solution}
      L'astuce ici est de prendre des nouvelles variables pour chacune des
      (in)équations.
      On réécrit donc le problème d'optimisation linéaire comme suit

      $
      \begin{array}{lrcr}
        \mini & c^T x\\
        & A x & = &  b\\
        & I x   & \geq & a\\
        & x & & \mbox{ libre}
      \end{array}
      $

      où le dual appairait plus clairement

      $
      \begin{array}{lrcr}
        \maxi & b^Ty_1+a^Ty_2\\
        & Ay_1 + Iy_2 & = & c\\
        & y_1 & & \text{libre}\\
        & y_2 & \geq & 0.
      \end{array}
      $

      Si $x$ était un vecteur, $y_1$ et $y_2$ seront des vecteurs également.

      On peut s'en sortir avec moins de variables $y_i$ en faisant
      le changement de variable $x' + a = x$, d'où

      $
      \begin{array}{lrcr}
        \mini & c^T x' + c^Ta\\
        & A x' & = &  b - Aa\\
        & x'   & \geq & 0.
      \end{array}
      $

      On peut retirer $c^Ta$ car c'est une constante.
      Le dual est alors

      $
      \begin{array}{lrcr}
        \maxi & (b-Aa)^Ty\\
        & A^Ty & \leq & c
      \end{array}
      $

      \textbf{TODO} comment utilise-t-on le fait que $a \geq 0$ ?

      \textbf{TODO} ça a du sens de prendre le dual si on a fait un changement
      de variable ?
    \end{solution}

  \item Trouvez une formulation duale du  problème

    $
    \begin{array}{lrcrcrcr}
      \mini & c^T x\\
      & b_1 & \leq & A x & \leq &  b_2\\
      & x   & \geq & 0
    \end{array}
    $

    \begin{solution}
      $
      \begin{array}{lrcrcrcr}
        \maxi & b_{1}^{T}y_{1} + b_{2}^{T}y_{2}\\
        & A^{T}y_{1} & \leq & c\\
        & A^{T}y_{2} & \geq & c\\
        & y_1 & \geq & 0\\
        & y_2 & \leq & 0
      \end{array}
      $
    \end{solution}


  \item Comment résoudre le problème

    $
    \begin{array}{llcr}
      \mini & 50x_1+25x_2\\
      & x_1+3x_2 & \geq & 8\\
      & 3x_1+4x_2 & \geq & 19\\
      & 3x_1+x_2 & \geq & 7\\
      & x_1, x_2 & \geq & 0
    \end{array}
    $

    sans effectuer de phase d'initialisation?


    \begin{solution}
      Dual:

      $
      \begin{array}{llcr}
        \maxi & 8y_{1} + 19y_{2} + 7y_{3}\\
        & y_1 + 3y_2 + 3y_3 & \leq & 50\\
        & 3y_1 + 4y_2 + y_3 & \leq & 25\\
        & y & \geq & 0
      \end{array}
      $

      Si $c \geq 0$, alors $y=0$ est une solution admissible de base du dual et on peut directement démarrer l'algorithme dual simple. Par la dualité forte, on aura au final $b^{T}y^{*} = c^{T}\xopt$.
    \end{solution}

  \item Soit le problème

    $
    \begin{array}{llcr}
      \mini & 2x_1-x_2\\
      & 2x_1-x_2 -x_3 & \geq & 3\\
      & x_1-x_2 +x_3 & \geq & 2\\
      & x_1, x_2, x_3 & \geq & 0
    \end{array}
    $

    Quel est le dual de ce problème?
    Proposez des bornes supérieures et inférieures aussi précises que possible pour l'objectif optimal du primal.
    Faites de même pour le dual.

    \begin{solution}

      $
      \begin{array}{llcr}
        \maxi & 3y_1 + 2y_2\\
        & 2y_1 + y_2 & \leq & 2\\
        & -y_1 - y_2 & \leq & -1\\
        & -y_1 + y_2 & \leq & 0\\
        & y_1, y_2 & \geq & 0
      \end{array}
      $

      Pour l'objectif du primal, on sait que que $z^* \geq 3$ car,
      par la première inéquation, on a $z = 2x_1 - x_2 \geq 3 + x_3$ et on
      a aussi par la dernière inéquation $x_3 \geq 0$.
      De plus, on remarque que $(2, 0, 0)$ est une solution admissible donnant
      $z = 2\cdot 2 - 0 = 4$ donc $3 \leq z^* \leq 4$.

      Pour l'objectif du dual, on a, par la première inéquation du dual,
      $$ z = 3y_1 + 2y_2 \leq 4y_1 + 2y_2 \leq 4 $$
      Comme $(1, 0)$ est une solution admissible du dual donnant $w = 3$,
      on a $3 \leq w^* \leq 4$.

      Pour les petits curieux, la solution optimal vaut $10/3$ et correspond
      à $(y_1, y_2) = (2/3, 2/3)$ et $(x_1, x_2, x_3) = (5/3, 0, 1/3)$.
    \end{solution}

  \item L'objectif optimal du problème

    $
    \begin{array}{llcr}
      \mini & 7x_1+10x_2\\
      & 2x_1+3x_2 & \geq & 10\\
      & 3x_1+4x_2 & \geq & 19\\
      & x_1+2x_2 & \geq & 9\\
      & x_1, x_2 & \geq & 0
    \end{array}
    $

    est égal à $z_*=47$. La solution $(0, 2, 1)$ est elle une solution admissible optimale du dual?


    \begin{solution}
      Dual:

      $
      \begin{array}{llcr}
        \max & 10y_{1} + 19y_{2} + 9y_{3}\\
        & 2y_{1} + 3y_{2} + y_{3} & \leq & 7\\
        & 3y_{1} + 4y_{2} + 2y_{3} & \leq & 10\\
        & y_{i} & \geq & 0
      \end{array}
      $

      La solution $y = (0,2,1)$ est admissible pour le problème dual car
      $y$ appartient au polyèdre.
      Et, par la dualité forte,
      comme $z^{*} = 47 = b^Ty$,
      c'est est une solution optimale
      car les coûts sont égaux pour les deux problèmes.
    \end{solution}

  \item Soit le problème suivant

    $
    \begin{array}{llcr}
      \mini & 2x_1+9x_2+3x_3\\
      & -2x_1+2x_2 +x_3 & \geq & 1\\
      & x_1+4x_2 -x_3 & \geq & 1\\
      & x_1, x_2, x_3 & \geq & 0
    \end{array}
    $

    Trouvez le dual de ce problème et résolvez-le graphiquement. Utilisez les relations d'exclusion pour obtenir une solution du primal.

    \begin{solution}
      La solution admissible optimal du dual est
      $y^{*} = (\frac{7}{2}, \frac{1}{2})$.
      La relation d'exclusion stipule que $x^{T}(c-A^{T}y) = 0$
      ce qui entraîne la condition $x_{1} = 0$.
      Pour minimiser le problème du primal,
      il suffit de serrer les contraintes pour $x_{2}$ et $x_{3}$.
      On obtient $\xopt = (0, \frac{1}{3}, \frac{1}{3})$.
    \end{solution}

  \item Soit le problème suivant

    $
    \begin{array}{llcr}
      \mini & 5x_1-3x_2\\
      & 2x_1-x_2 +4x_3 & \leq & 4\\
      & x_1+x_2 +2x_3 & \leq & 5\\
      & 2x_1-x_2 +x_3 & \geq & 1\\
      & x_1, x_2, x_3 & \geq & 0
    \end{array}
    $

    Le sommet associé à la base $(x_1, x_2, x_3)$ est un sommet optimal de ce problème. Quel est le dual de ce problème? Trouvez une solution du problème dual.


    \begin{solution}
      Dual:

      $
      \begin{array}{llcr}
        \max & -4y_{1} - 5y_{2} + y_{3}\\
        & -2y_{1} - y_{2}  + 2y_{3} & \leq & 5\\
        & y_{1} - y_{2}  -y_{3} & \leq & -3\\
        & -4y_{1} - 2y_{2}  + y_{3} & \leq & 0\\
        & y & \geq & 0
      \end{array}
      $

      Comme $x_1, x_2, x_3$ sont dans la base, aucun n'est nul.
      Comme c'est un \emph{sommet} optimal, il faut que 3 contraintes
      soient actives, ça doit donc être les 3 premières.
      Il nous faut donc résoudre le système $Ax = b$
      \[
        \begin{pmatrix}
          2 & -1 & 4\\
          1 & 1 & 2\\
          2 & -1 & 1
        \end{pmatrix}
        \begin{pmatrix}
          x_{1}\\
          x_{2}\\
          x_{3}
        \end{pmatrix}
        \stackrel{=}{}
        \begin{pmatrix}
          4\\
          5\\
          1
        \end{pmatrix}
      \]
      ce qui nous donne
      $\xopt = (1,2,1)$ comme solution du primal.
      Or, par la relation d'exclusion, $x^{*T}(c -A^{T}y^*) = 0$.
      Dans ce cas, il faut résoudre le système $A^Ty = c$
      \[
        \begin{pmatrix}
          -2 & -1 & 2\\
          1 & -1 & -1\\
          -4 & -2 & 1
        \end{pmatrix}
        \begin{pmatrix}
          y_{1}\\
          y_{2}\\
          y_{3}
        \end{pmatrix}
        \stackrel{=}{}
        \begin{pmatrix}
          5\\
          -3\\
          0
        \end{pmatrix}
      \]
      %Comme $\det(A) \neq 0$, la solution est unique.
      ce qui donne $y = \frac{1}{3}(2, 1, 10)$.
      En effet, on retrouve bien que $c^T y = -1$ et que $y \geq 0$,
      la dualité forte nous indique donc que $y$ est une solution optimale
      bien que la relation d'exclusion nous le disait déjà.
    \end{solution}

  \item  Jacques choisit un nombre de 1 à $m$, son adversaire, Yvan, choisit indépendamment
    de Jacques, un nombre de 1 à $n$. Si Jacques choisit $i$ et Yvan $j$, Jacques reçoit
    $a_{ij}$ d'Yvan. Jacques et Yvan appliquent des stratégies probabilistes; Jacques
    choisit
    $i$ avec une probabilité
    $x_i$  ($x_i \geq 0$ et $\sum_1^m x_i =1$) et Yvan choisit $j$ avec une probabilité
    $y_j$ ($y_j \geq 0$ et $\sum_1^n y_i =1$).

    \begin{enumerate}
      \item Soit $x$ et $y$ les stratégies de Jacques et d'Yvan. Quelle est le gain moyen de
        Jacques par partie?
      \item Jacques divulgue la stratégie probabiliste qu'il applique. Quelle stratégie Yvan
        doit-il appliquer pour maximiser son gain? Formulez ce problème comme un problème
        d'optimisation.
      \item \label{l1} Nous considérons maintenant un jeu précis. Jacques et Yvan choisissent tout les deux un nombre
        de  1 à 3. Le joueur ayant choisi le nombre le plus élevé reçoit 1 EUR de son adversaire, sauf si ce
        nombre est exactement d'une unité supérieur au nombre choisi par l'adversaire, auquel cas,
        il donne 3 EUR à son adversaire. Lorsque les nombres choisis sont égaux, les joueurs font
        match nul et personne ne gagne rien. Jacques annonce sa stratégie probabiliste: il joue 1,
        2 et 3 avec les probabilités $0$,
        $1/2$ et $1/2$. Quelle stratégie Yvan doit-il choisir pour maximiser son gain et quel est
        son gain moyen par partie? La stratégie optimale pour Yvan est-elle unique?
      \item \label{t1} On considère le jeu décrit en \ref{l1}. Jacques sait que s'il joue suivant la stratégie
        probabilitiste
        $x$; Yvan choisira la stratégie probabiliste $y$ qui maximise son gain moyen. Sachant cela, quelle stratégie
        Jacques doit-il adopter pour maximiser son gain moyen? Ecrivez ce
        problème sous forme d'un problème d'optimisation linéaire.

      \item Résolvez le problème formulé en \ref{t1}. Quel est le gain moyen de Jacques?

    \end{enumerate}



    \begin{solution}
      \nosolution
    \end{solution}

  \item Soit le problème d'optimisation linéaire suivant:

    $
    \begin{array}{rrllll}
      \mini & \sum_{i=1}^n c_i x_i& & \\
      &\sum_{i=1}^n a_i x_i&=&b  \\
      &x_i  & \geq & 0 & i=1, \ldots, n
    \end{array}
    $



% polyhedron

    Remarquez que ce problème ne possède qu'une seule contrainte.

    \begin{enumerate}
      \item  En utilisant un argument élémentaire, trouvez une condition simple d'existence d'une solution admissible.


      \item  En  supposant que le coût optimal est fini, proposez une méthode d'obtention d'une solution optimale et justifiez votre réponse. (Indice: pour répondre à cette question vous pouvez si vous le désirez faire appel au théorème fondamental de l'optimisation linéaire.)

      \item Sous quelles conditions le problème possède-t-il plusieurs solutions?

    \end{enumerate}



    \begin{solution}
      \begin{enumerate}
        \item Au moins un des $a_i$ doit être de même signe que $b$.
        \item Dans le cas où on a un sommet, comme le coût optimal est fini,
          par le théorème fondamental,
          on sait qu'il existe un \emph{sommet} optimal.
          Comme c'est un sommet, au moins $n$ contraines sont actives.
          Il y a donc au moins $n-1$ $x_i$ qui sont nuls.
          La dernière variable faut alors $\frac{b}{a_i}$.
          On peut donc trouver une solution optimale en prenant,
          parmis toutes les soltions tels que $\frac{b}{a_i} \geq 0$,
          celle qui a
          \[ c_i \frac{b}{a_i} \]
          le plus petit.
        \item S'il existe $i \neq j$ tel que
          \[ z^* = \frac{c_i}{a_i} = \frac{c_j}{a_j} \]
          et que $b \neq 0$ (sinon ces deux solutions sont les mêmes
          avec $x = 0$),
          alors le problème possède plusieurs solutions optimales.
      \end{enumerate}
    \end{solution}

  \item Soit le problème d'optimisation
    \begin{eqnarray*}
      \max_{x_i} & x_1+ x_2+ x_3 &  \\
      &2 x_1 + x_3  &\leq 2 \\
      &2 x_2 + x_3  &\leq 2 \\
      &x_i &\geq 0
    \end{eqnarray*}

    \begin{enumerate}
      \item Enumérez les sommets de ce polyèdre. L'un des
        ces sommets est-il optimal?
      \item Quelles sont (toutes) les
        solutions optimales?
      \item Que deviennent ces solutions optimales si
        on exige que les solutions soient entières?
      \item Ecrivez le dual de
        ce problème, résolvez-le, et faites le lien avec la solution du
        primal.
    \end{enumerate}

    \begin{solution}
      \begin{enumerate}
        \item On a 3 variable, il faut donc serrer 3 contraintes.
          En choisissant $x_1 = 0$, on a $(0, 0, 2)$ qui est un sommet
          dégénéré car il sert toutes les contraintes sauf $x_3 \geq 0$.
          En choisissant $x_3 = 0$, on a $(1, 1, 0)$.

          Le polyèdre possède donc manifestement au moins un sommet et
          en prenant la moitié de la première contrainte plus la moitié
          de la seconde, on trouve $x_1 + x_2 + x_3 \leq 2$.
          Le coût est donc fini.
          Par le théorème fondamental, il existe donc un sommet optimal.
          Un des deux sommets est donc optimal.
          On remarque le coût optimal est le même pour les deux sommets
          sommets $z^* = 2$.
          Les deux sommets sont donc des solutions optimales.
        \item L'ensemble des solutions optimales est convexe donc
          l'ensemble des solutions optimales est donné pour chaque
          $0 \leq \lambda \leq 1$ par
          \[ \lambda(0,0,2) + (1-\lambda)(1,1,0). \]
        \item Le coût optimal reste 2 car $(1,1,0)$ est entier mais il faut
          que $(1-\lambda,1-\lambda,2\lambda)$ soit entier.
          Il n'y a donc plus que les deux sommets comme solution optimale.
        \item Le dual vaut
          \begin{align*}
            \min_{x_i}  2y_1 + 2y_2\\
            2y_1 & \geq 1\\
            2y_2 & \geq 1\\
            y_1 + y_2 & \geq 1\\
            y & \geq 0
          \end{align*}
          On voit tout de suite que la troisième contrainte nous donne
          une borne inférieure pour le coût optimal de 2.
          On a cette fois-ci un seul sommet, $(1/2,1/2)$.
          On sait donc par le théorème fondamental que c'est la solution
          optimale.
          On trouve donc un coût optimal de 2 ce qui était prévisible
          par la dualité forte.

          On voit aussi que aucun $y_i$ n'est nul ce qui par la relation
          d'exclusion implique que les deux contraintes du primal
          doivent être serrées. C'est bien ce qu'on avait obtenu.
      \end{enumerate}
    \end{solution}

  \item Soit le problème d'optimisation
    \begin{eqnarray*}
      \min_{x_i} & 3 x_1- x_2+ x_3 &  \\
      &-x_1 -x_2+ x_3  &\ge 1 \\
      &x_1- x_2 +\alpha x_3  &= 2 \\
      &x_1, x_2 &\ge 0
    \end{eqnarray*}
    où  $0.25\le\alpha\le4$ est un paramètre.
    \begin{enumerate}
      \item Quel est le dual de ce problème?
      \item Résolvez le dual en fonction
        du paramètre $\alpha$ par une méthode graphique.
      \item Déduisez les solutions du primal.
    \end{enumerate}


    \begin{solution}
      \begin{enumerate}
        \item
          \begin{align*}
            \max y_1 + 2y_2\\
            -y_1 + y_2 & \leq 3\\
            -y_1 - y_2 & \leq -1\\
            y_1 + \alpha y_2 & = 1\\
            y_1 & \geq 0\\
            y_2 & \text{ libre}.
          \end{align*}
        \item
          La figure ci-dessous représente en hashuré le domaine admissible
          sans prendre compte la contrainte $y_1 + \alpha y_2 = 1$.
          \begin{center}
            \begin{tikzpicture}
              % Axis
              \draw[axis] (-1,0) -- (1,0) node[right=\nudge cm] {\(y_1\)};
              \draw[axis] (0,0) -- (0,3) node[above=\nudge cm] {\(y_2\)};
              \begin{scope}%je t'aime mon benoit
                % Avoid going too far
                \clip (-1-\nudge,-\nudge) rectangle (1+\nudge,3.3+\nudge);
                \draw[line] (-1,2) -- (1,4) coordinate (ineq1);
                \draw[line] (-1,2) -- (1.3,-0.3) coordinate (ineq2);
                \draw[line] (1,0) -- (1/5,16/5) coordinate (ineq3);
                \draw[line] (2,-1/4) -- (-1,1/2) coordinate (ineq4);
                \begin{scope}
                  %Clip sélectionne ce qu'il y a dans le polygone dessiné.
                  \clip (0,1) -- (0,3) -- (2,5) |- (2,-1);
                  \fill[hash] (-1,-0.5) rectangle (1.4,3.4);
                  %\draw[dashed,line] (0,1) -- (2,2) -- (3,1) -- (2,0);
                  % (4,4) to be enough, anyway clip restrict us to the good part
                \end{scope}
              \end{scope}
              %\node[above right] at (ineq1) {\(\mathllap{-}3 x_1 + 4 x_2 = 4\)};
              \node[right] at (ineq1) {\(\mathllap{-}y_1 + y_2 = 3\)};
              \node[right] at (ineq2) {\(\mathllap{-}y_1 - y_2 = -1\)};
              \node[right] at (ineq3) {\(y_1 + 0.25y_2 = 1\)};
              \node[above] at (ineq4) {\(y_1 + 4y_2 = 1\)};
              \foreach \coord/\adj in {
                {(-1,2)}/left,
                {(0,3)}/left,
                %{(100,50)}/above right,
                %{(0,200/3)}/left,
              } {
                \fill \coord circle (2pt) node[\adj] {$\coord$};
              }
            \end{tikzpicture}
          \end{center}
          \begin{itemize}
            \item Lorsque $\alpha \leq 1/3$, c'est l'intersection entre
              $y_1 + \alpha y_2 = 1$ et $-y_1 + y_2 = 3$  qui est optimale;
            \item Lorsque $1/3 < \alpha \leq 1$, c'est l'intersection entre
              $y_1 + \alpha y_2 = 1$ et $y_2 = 0$  qui est optimale;
            \item Lorsque $1 < \alpha < 2$, il reste plus qu'un point dans le
              domaine, $(1,0)$. C'est donc la solution optimale.
            \item Lorsque $\alpha = 2$, il y a une infinité de solutions
              de coût optimal 1 qui définissent l'ensemble
              \[ \{(y_1,(1-y_1)/2) | y_1 \geq 0\}. \]
            \item Lorsque $\alpha > 2$, le coût optimal est non borné.
          \end{itemize}
        \item
          \begin{itemize}
            \item Lorsque $\alpha \leq 1/3$, la deuxième contrainte
              du dual est inactive donc,
              par la relation d'exclusion, $x_2 = 0$, on a donc
              $x_1 = 2 - \alpha x_3$.
              \begin{itemize}
                \item Si $\alpha \neq 1/3$, $y_2 > 0$ du coup,
                  par la relation d'exclusion, $-x_1 -x_2+ x_3 = 1$ ce qui
                  donne $x^* = \frac{1}{\alpha+1}(2-\alpha,0, 3)$.
                \item Si $\alpha = 1/3$, la fonction objectif devient
                  $\min 2$ et on a une solution optimale pour tout
                  $\frac{9}{4} \leq x_3 \leq 6$ et donc
                  $0 \leq x_1 \leq \frac{5}{4}$.
              \end{itemize}
            \item Lorsque $1/3 < \alpha < 1$,
              aucune des 2 premières contraintes du dual ne sont actives.
              Dès lors,
              par la relation d'exclusion, $(x_1,x_2) = 0$ et donc
              $x^* = \frac{1}{\alpha}(0,0,2)$.
            \item Lorsque $\alpha = 1$, la première
              contrainte qui n'est pas active et donc $x_1 = 0$.
              La deuxième contrainte du primal nous donne alors
              $-x_2 + x_3 = 2$ qui avec une fonction objectif $-x_2 +x_3$
              nous indique que $z^* = 2$.
              L'ensemble des solutions optimales est donc
              \[ \{(0,x_2,2-x_2) | x_2 \geq 0 \}. \]
            \item Lorsque $1 < \alpha < 2$, la première contrainte du dual
              n'est pas active et $y_1 \geq 0$ non plus.
              On a donc $x_1 = 0$ et $-x_1-x_2+x_3 = 1$.
              En résolvant, on obtient
              $x^* = \frac{1}{\alpha-1}(0,2-\alpha,1)$.
            \item Lorsque $\alpha = 2$, comme la première contrainte
              du dual est inactive, $x_1 = 0$.
              En prenant 2 fois la premières contrainte du primal
              moins la deuxième, on tombe sur $-x_2 \geq 0$.
              Comme on a aussi que $x_2 \geq 0$, on a $x_2 = 0$ et
              on a comme solution optimale $(0,0,1)$ avec comme coût
              optimal 1 comme pour le dual.
            \item Lorsque $\alpha > 2$, comme le coût optimal du dual
              est non-borné, le primal ne possède pas de solution
              admissible.
          \end{itemize}
      \end{enumerate}
    \end{solution}

  \item Une entreprise vend 20 tonnes de sa production à répartir
    entre 5 acheteurs (voir les prix payés dans le tableau plus bas).
    Pour des raisons de logistique:
    \begin{itemize}
      \item l'acheteur A a un accord privilégié et achète au moins 2
      tonnes, dans toutes les circonstances. \item l'acheteur B achète
      au plus 2 tonnes. \item l'acheteur C achète au moins 2 tonnes ou
      alors n'achète rien. \item l'acheteur D achète au plus 5 tonnes.
      \item l'acheteur E achète une quantité qui ne diffère pas de plus
        de 3 tonnes avec la quantité achetée par D.
    \end{itemize}

    Les prix payés par les différents acheteurs sont, en milliers
    d'Euros par tonne:
    \begin{center}
      \begin{tabular}{|c|c|c|c|c|c|}
        \hline %\vspace{2pt}
        Acheteur & A & B & C & D & E\\
        \hline
        Prix payé & 20 & 50 & 20 & 25 & 15 \\
        \hline
      \end{tabular}
    \end{center}

    \begin{enumerate}
      \item  Formulez ce problème comme un problème d'optimisation. Ce
        problème est-il linéaire ? Sinon, reformulez-le comme un problème
        d'optimisation linéaire.
%%\\(b) Sans calculer la solution optimale, que pouvez-vous dire du nombre
%%de contraintes serrées à l'optimal ? Cela vous permet-il d'obtenir
%%des informations au sujet de la solution optimale ? Si oui,
%%lesquelles ? Sinon, pourquoi ? Justifiez votre réponse.

      \item %%Supposez que la résolution du problème formulé en (a) vous fournisse un
%%objectif optimal de valeur $z^*$.
        Formulez le problème qui consisterait à maximiser la quantité
        vendue à l'acheteur C, parmi l'ensemble des solutions optimales du
        problème formulé en (a).
      \item  Résolvez le problème formulé en (a), en considérant que
        l'acheteur C n'achète rien.

    \end{enumerate}


    \begin{solution}
      \nosolution
    \end{solution}

  \item Vrai ou faux? Justifiez vos choix par quelques lignes, un contre-exemple ou un dessin. Soyez assez précis pour convaincre
    que vous ne devinez pas la réponse mais ne fournissez toutefois pas une justification formelle et détaillée.

    \begin{enumerate}



% duality

      \item Le dual d'un problème d'optimisation linéaire existe toujours.

%\item Un problème d'optimisation linéaire  possède exactement autant de variables que son dual.

      \item Si un problème d'optimisation linéaire possède une solution admissible, alors son dual  également et les coûts optimaux sont égaux.

      \item Si un problème d'optimisation linéaire admet un coût optimal fini, alors son dual également.

%\item If a linear optimization problem has an unbounded optimal
%cost, then its dual is infeasible.


%\item If a linear optimization problem is infeasible, then its
%dual has unbounded optimal cost.


%\item L'optimisation linéaire permet de faire une régression linéaire minimisant la norme $||\cdot||_\infty$ des résidus.




    \end{enumerate}


    \begin{solution}
      \begin{enumerate}
        \item Vrai.
        \item Faux, on peut avoir un objectif non-borné du primal et donc
          pas de solution admissible du dual et inversément.
        \item Vrai, par la dualité faible, et dualité forte.
      \end{enumerate}
    \end{solution}


\end{enumerate}
