\section{Méthode du simplexe}

\begin{enumerate}

  \item Soit le problème d'optimisation

    $
    \begin{array}{lrcr}
      \mini & -2x_1 - x_2 \\
      & x_1 - x_2 & \leq & 2\\
      & x_1 +  x_2 & \leq & 6\\
      & x_1, x_2 & \geq & 0
    \end{array}
    $

    Convertissez ce problème sous forme standard et trouvez un sommet pour lequel $x_1 =x_2=0$. Résolvez le problème au moyen de la méthode
    du simplexe. Tracez une représentation graphique en terme des variables $x_1, x_2$ et indiquez le chemin suivi par la méthode.

    \begin{solution}
      $x^{*} = (4,2,0,0)$
    \end{solution}

  \item Considérez le problème



    $
    \begin{array}{llcr}
      \mini & 20 x_1+\alpha x_2+12 x_3\\
      & x_1  & \leq & 400\\
      &2x_1+\beta x_2+x_3 & \leq & 1000\\
      &2x_1+\gamma x_2+ 3x_3 & \leq & 1600\\
      & x_1, x_2, x_3 & \geq & 0
    \end{array}
    $

    Proposez, si possible, des valeurs pour $\alpha, \beta$ et $\gamma$ pour lesquelles:

    \begin{enumerate}

      \item Le coût optimal est fini et la solution optimale est unique.

      \item Le coût optimal est fini et il y a une infinité de solutions optimales.

      \item Le coût optimal est non borné (trouvez une paramétrisation de valeurs de $x$ parmi lesquelles se trouvent des solutions de coûts
        arbitrairement faibles).

      \item Le poly\`edre possède un sommet dégénéré.




    \end{enumerate}







    \begin{solution}
      Néant
    \end{solution}

  \item
    Résoudre par l'algorithme du simplexe les problèmes



    $
    \begin{array}{llcr}
      \maxi & 2x_1+3x_2\\
      & x_1+2x_2 & \leq & 4\\
      & x_1+x_2 & = & 3\\
      & x_1, x_2& \geq & 0
    \end{array}
    $



    $
    \begin{array}{llcr}
      \maxi & 20 x_1+16x_2+12 x_3\\
      & x_1  & \leq & 400\\
      &2x_1+x_2+x_3 & \leq & 1000\\
      &2x_1+2x_2+ 3x_3 & \leq & 1600\\
      & x_1, x_2, x_3 & \geq & 0
    \end{array}
    $



    $
    \begin{array}{llcr}
      \maxi & 4x_1+3x_2+6x_3\\
      & 3x_1+x_2+3x_3 & \leq & 30\\
      & 2x_1+2x_2+3x_3 & \leq & 40\\
      & x_1, x_2, x_3 & \geq & 0
    \end{array}
    $





    $
    \begin{array}{llcr}
      \maxi & -x_1+4x_2\\
      & -3x_1+4x_2 & \leq & 6\\
      & x_1+2x_2 & \leq & 4\\
      & x_2& \geq & -3
    \end{array}
    $








    \begin{solution}
      a) $x^{*} = (2,1,0)$ \\
      \newline
      $\begin{array}
      {ccc|l}
      0 & 0 & 1 & -z+7\\ \hline
      1 & 0 & -1 & 2 \\
      0 & 1 & 1 & 1
      \\\end{array}$ \\
      \newline
      \newline
      b) $ x^{*} = (200,600,0,300,0,0)$ \\
      \newline
      $\begin{array}
      {cccccc|l}
      0 & 0 & 10 & 0 & 4 & 6 & -z + 13600 \\  \hline
      1 & 0 & -0.5 & 0 & 1 & -0.5 & 200 \\
      0 & 1 & 2 & 0 & -1 & 1 & 600 \\
      0 & 0 & 0.5 & 1 & -1 & 0.5 & 200
      \\\end{array}$\\
      \newline
      \newline
      c) $x^{*} = (0,10,\frac{20}{3}, 0,0)$ \\
      \newline
      $\begin{array}
      {ccccc|l}
      1 & 0 & 0 & \frac{5}{2} & 1 & -z +70 \\ \hline
      \frac{4}{3} & 0 & 1 & \frac{2}{3} & -\frac{1}{3} & \frac{20}{3} \\
      -1 & 1 & 0 & - \frac{1}{2} & 1 & 10
      \\\end{array}$\\
      \newline
      \newline
      d) $x^{*} = (\frac{2}{5}, \frac{9}{5}, 0,0,\frac{24}{5})$ \\
      \newline
      $\begin{array}
      {ccccc|l}
      0 & 0 & \frac{3}{5} & \frac{4}{5} & 0 & - z + \frac{34}{5} \\ \hline
      0 & 1 & \frac{1}{10} & \frac{3}{10} & 0 & \frac{9}{5} \\
      1 & 0 & -\frac{1}{5} & \frac{2}{5} & 0 & \frac{2}{5} \\
      0 & 0 & \frac{1}{10} & \frac{3}{10} & 1 & \frac{24}{5}
      \\\end{array}$\\
      \newline
      \newline
    \end{solution}

  \item Nous reprenons un des problèmes précédents. Un étudiant dispose de 100 heures de travail pour
    étudier les examens A, B et C. Il pense gagner par heure de travail sur chaque cours 1/5 de points pour
    le cours A, 2/5 de points pour le cours B et 3/5 pour le cours C. Chaque examen est coté sur 20. Les
    exercices de ces cours comptent pour la moitié de la cote finale. Ses résultats pour les exercices lui
    ont été communiqués. Il a obtenu 12/20 pour A, 12.5/20 pour B et 13.4/20 pour C.  L'étudiant doit obtenir au minimum une cote globale de 10/20 pour chaque
    cours. Tous les cours ont la même pondération et l'étudiant désire obtenir la   moyenne la plus élevée possible.

    Formulez ce problème comme un problème d'optimisation linéaire  et résolvez-le. Vous pouvez utiliser le fait que l'étudiant a avantage à utiliser la
    totalité des 100 heures de travail. L'étudiant obtiendra-t-il une distinction?





    \begin{solution}
      Soient $t_{A}$, $t_{B}$ et $t_{C}$, le nombre d'heures étudiées pour $A$, $B$ et $C$.
      \newline
      $$ \frac{37.9}{120} -\min - \frac{1}{120}(t_{A} + 2t_{B} + 3t_{C}) $$
      avec les contraintes : \\
      \newline
      $40 \le t_{A} \le 140$ \\
      $18.75 \le t_{B} \le 68.75 $ \\
      $11 \le t_{C} \le 43.3333$ \\
      $t_{A} + t_{B} + t_{C} = 100$ \\
      \newline
      Le tableau initial est donné par : \\
      \newline
      $\begin{array}
      {ccccccccc|l}
      -\frac{1}{120} & -\frac{2}{120} & -\frac{3}{120} &0&0&0&0&0&0& -z + 0.316\\ \hline
      1&0&0&1&0&0&0&0&0&140 \\
      1&0&0&0&-1&0&0&0&0&40 \\
      0&1&0&0&0&1&0&0&0&68.75 \\
      0&1&0&0&0&0&-1&0&0&18.75 \\
      0&0&1&0&0&0&0&1&0&43.3333 \\
      0&0&1&0&0&0&0&0&-1&11 \\
      1&1&1&0&0&0&0&0&0&100
      \\\end{array}$\\
      \newline
      \newline
      Solution : $T^{*} = (40,18.75,41.25,\dots) \in \mathbb{R}^{9}$ soit une moyenne de $65,13$\% (Satisfaction).
    \end{solution}

  \item Une société produit des biens A, B et C. La production des biens nécessite l'utilisation de 4 machines. Les temps de
    production et les profits générés sont repris dans le tableau\\

    $
    \begin{array}{l|llll|l}
      & 1 & 2 & 3 & 4 & \mbox{profit}\\
      \hline
      A & 1 & 3 & 1 & 2 & 6\\
      B & 6 & 1 & 3 & 3 & 6\\
      C & 3 & 3 & 2 & 4 & 6
    \end{array}
    $
    \\

    Si les temps de production disponibles sur les machines 1, 2, 3 et 4  sont de 84, 42, 21 et 42,
    déterminez la quantité de biens à produire pour maximiser le profit.






    \begin{solution}
      Soit $x_{ij}$ la quantité de biens $i$ produite par la machine $j$ et $t_{ij}$, le temps de production du bien $i$ par la machine $j$.
      \newline
      $$ -\min -6\sum_{i}^{3} \sum_{j}^{4}x_{ij}$$
      sous les contraintes \\
      $$ \sum_{i}^{3} t_{i1}x_{i1} \le 84$$
      $$ \sum_{i}^{3} t_{i2}x_{i2} \le 42$$
      $$ \sum_{i}^{3} t_{i3}x_{i3} \le 21$$
      $$ \sum_{i}^{3} t_{i4}x_{i4} \le 42$$
      Le tableau final est :\\
      \newline
      $\begin{array}
      {cccccccccccccccc|l}
      0&12&0&0&30&0&12&3&12&12&6&6&10&6&0&4.5&-z+1008 \\ \hline
      1&0&0&0&6&0&0&0&3&0&0&0&1&0&0&0&84\\
      0&3&0&0&0&1&0&0&0&3&0&0&0&1&0&0&42\\
      0&0&1&0&0&0&3&0&0&0&2&0&0&0&1&0&21\\
      0&0&0&1&0&0&0&\frac{3}{2}&0&0&0&2&0&0&0&1&21
      \\\end{array}$\\
    \end{solution}

  \item Résoudre par la méthode du simplexe en utilisant la règle de Bland



    $
    \begin{array}{llcr}
      \maxi & 10 x_1-57 x_2-9x_3-24x_4\\
      & 0.5x_1-5.5x_2-2.5x_3+9x_4 & \leq & 0\\
      & 0.5x_1-1.5x_2-0.5x_3+x_4 & \leq & 0\\
      & x_1 & \leq &1\\
      & x_1, x_2, x_3, x_4 & \geq & 0
    \end{array}
    $


    \begin{solution}
      La règle de Bland consiste à choisir la variable $x_{r}$ de plus petit indice $r$ parmi les variables candidates à l'entrée $i = 1,2, \dots, n$ (idem pour les variables candidates à la sortie). \\
      \newline
      Néant
    \end{solution}

  \item Proposez une méthode de recherche d'un sommet du polyhèdre

    $
    \begin{array}{lrcr}
      & 2x_1-3x_2 +2x_3 & \geq & 3\\
      & -x_1+x_2 +x_3 & \geq & 5\\
      & x_1, x_2, x_3 & \geq & 0
    \end{array}
    $

    \begin{solution}
       Le tableau d'initialisation est : \\
       \newline
      $\begin{array}
      {ccccccc|l}
      0&0&0&0&0&1&1&z  \\ \hline
      2&-3&2&-1&0&1&0&5 \\
      -1&1&1&0&-1&0&1&5
      \\\end{array}$\\
      \newline
      \newline
      $x^{*} = (0,0,0,0,0,5,5)$ est un sommet de ce polyèdre.
    \end{solution}

  \item  Considérez le problème d'optimisation

    $
    \begin{array}{llcr}
      \mini & 20 x_1+\alpha x_2+12 x_3\\
      & x_1  & \leq & 4\\
      &2x_1 - x_2+x_3 & \leq & 10\\
      &2x_1+\beta x_2+ 3x_3 & \leq & 16\\
      & x_1, x_2, x_3 & \geq & 0
    \end{array}
    $


    Trouvez une solution optimale au moyen de l'algorithme du simplexe lorsque $\alpha=-2$ et $\beta =1$. Proposez des
    valeurs pour
    $\alpha$ et $\beta$  pour lesquelles le coût optimal est non borné et proposez dans ce cas une solution pour laquelle le coût
    optimal est inférieur à -1000.





    \begin{solution}
      Pour $\alpha = -2$ et $\beta = 1$, le tableau final devient : \\
       \newline
      $\begin{array}
      {cccccc|l}
      24&0&18&0&0&2&z+32 \\ \hline
      1&0&0&1&0&0&4 \\
      4&0&4&0&1&0&26 \\
      2&1&3&0&0&1&16
      \\\end{array}$\\
      \newline
      La solution vaut $x^{*} = (0,16,0,4,26,0)$. \\
      \newline
      Le coût est non-borné pour $\alpha$ et $\beta$ strictement négatifs. (Voir CM4, tr. 16) \\
      Pour obtenir un coût inférieur à -1000, il faut paramétrer la demi-droite totalement contenue dans le polyèdre : \\
      On pose $x_{2} = \lambda \geq 0$. Dans ce cas, la demi-droite $(0,\lambda,,0,4,10+\lambda,16-\beta \lambda)$ appartient au polyèdre. On a donc $z = \alpha \lambda$. Pour obtenir un coût inférieur à -1000, on peut prendre par exemple : \\
      \newline
      $\lambda = 1$ \\
      $\beta = -1$ \\
      $\alpha = -2000$
    \end{solution}

  \item  Considérez le problème d'optimisation

    $
    \begin{array}{llcr}
      \maxi & \alpha x_1+16x_2+12 x_3\\
      & x_1  & \leq & 400\\
      &2x_1+x_2+x_3 & \leq & 1000\\
      &2x_1+2x_2+ 3x_3 & \leq & 1600\\
      & x_1, x_2, x_3 & \geq & 0
    \end{array}
    $

    Trouvez une solution optimale au moyen de l'algorithme du simplexe lorsque $\alpha=20$.  Existe-t-il une valeur de $\alpha$ pour laquelle le coût optimal
    est non borné? Si oui,  proposez une solution pour laquelle le coût  est supérieur à 10000. Si non, justifiez votre réponse.

    \begin{solution}
      Même principe.
    \end{solution}

\end{enumerate}
